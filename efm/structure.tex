%%%%%%%%%%%%%%%%%%%%%%%%%%%%%%%%%%%%%%%%%
% Arsclassica Article
% Structure Specification File
%
% This file has been downloaded from:
% http://www.LaTeXTemplates.com
%
% Original author:
% Lorenzo Pantieri (http://www.lorenzopantieri.net) with extensive modifications by:
% Vel (vel@latextemplates.com)
%
% License:
% CC BY-NC-SA 3.0 (http://creativecommons.org/licenses/by-nc-sa/3.0/)
%
%%%%%%%%%%%%%%%%%%%%%%%%%%%%%%%%%%%%%%%%%

%----------------------------------------------------------------------------------------
%	REQUIRED PACKAGES
%----------------------------------------------------------------------------------------

\usepackage[
nochapters, % Turn off chapters since this is an article        
beramono, % Use the Bera Mono font for monospaced text (\texttt)
eulermath,% Use the Euler font for mathematics
pdfspacing, % Makes use of pdftex’ letter spacing capabilities via the microtype package
dottedtoc % Dotted lines leading to the page numbers in the table of contents
]{classicthesis} % The layout is based on the Classic Thesis style

\usepackage{arsclassica} % Modifies the Classic Thesis package

\usepackage[T1]{fontenc} % Use 8-bit encoding that has 256 glyphs

%\usepackage[utf8]{inputenc} % Required for including letters with accents
\usepackage[latin1, utf8]{inputenc}

\usepackage[spanish, activeacute]{babel}

\usepackage{graphicx} % Required for including images
\graphicspath{{Figures/}} % Set the default folder for images

\usepackage{enumitem} % Required for manipulating the whitespace between and within lists

\usepackage{subfig} % Required for creating figures with multiple parts (subfigures)

\usepackage{amsmath,amssymb,amsthm} % For including math equations, theorems, symbols, etc

\usepackage{varioref} % More descriptive referencing

\usepackage{titling}

\usepackage{multicol}
\usepackage{multirow}
\usepackage{booktabs}
\usepackage{tabularx}
\usepackage{array}
\usepackage{float}

\usepackage{cite} % para contraer referencias

\usepackage{enumitem}

\usepackage{makeidx}

%----------------------------------------------------------------------------------------
%	THEOREM STYLES
%---------------------------------------------------------------------------------------

\theoremstyle{definition} % Define theorem styles here based on the definition style (used for definitions and examples)
\newtheorem{definition}{\color{blue}{Definición}}

\theoremstyle{plain} % Define theorem styles here based on the plain style (used for theorems, lemmas, propositions)
\newtheorem{theorem}{Teorema}[section]

\theoremstyle{remark} % Define theorem styles here based on the remark style (used for remarks and notes)

%----------------------------------------------------------------------------------------
%	HYPERLINKS
%---------------------------------------------------------------------------------------
\definecolor{blue(pigment)}{rgb}{0.0, 0.28, 0.67}

\usepackage{hyperref}
\hypersetup{
%draft, % Uncomment to remove all links (useful for printing in black and white)
colorlinks=true, breaklinks=true, bookmarks=true,bookmarksnumbered,
urlcolor=blue(pigment), linkcolor=blue(pigment), citecolor=webgreen, % Link colors
pdftitle={}, % PDF title
pdfauthor={\textcopyright}, % PDF Author
pdfsubject={}, % PDF Subject
pdfkeywords={}, % PDF Keywords
pdfcreator={pdfLaTeX}, % PDF Creator
pdfproducer={LaTeX with hyperref and ClassicThesis} % PDF producer
}

%----------------------------------------------------------------------------------------
%	BACKGROUND IMAGE
%----------------------------------------------------------------------------------------

\usepackage{graphicx} \usepackage{eso-pic}

\newcommand\BackgroundPic{\put(85,50){ \parbox[b][\paperheight]{\paperwidth}{ \vfill \centering \includegraphics[width=8cm,keepaspectratio]{portada.jpg} \vfill }}}


%----------------------------------------------------------------------------------------
%	MARGIN
%----------------------------------------------------------------------------------------

\usepackage{vmargin}
\usepackage{parskip}
\setlength{\parindent}{1.5em}
\setmargins
	{3cm}		%izq
	{1.5cm}		%sup
	{15.5cm}	%anchura texto
	{22cm}		%altura texto
	{25pt}		%altura encabezados
	{1.5cm}		%espacio entre texto y encabezados
	{60pt}		%altura pie de pagina
	{1.5cm}		%espacio entre texto y pie de pagina

%----------------------------------------------------------------------------------------
%	TITLES
%----------------------------------------------------------------------------------------

%\titleformat{\section}{\textbf}{}{0cm}{}
%\titleformat{\subsection}{\textbf{\thesubsection.\thesubsectionname}}{}{0cm}{}

%----------------------------------------------------------------------------------------
%	PAGINA HORIZONTAL
%----------------------------------------------------------------------------------------

\usepackage{lscape,lipsum}

\usepackage{pdflscape}

%----------------------------------------------------------------------------------------
%	APENDICES
%----------------------------------------------------------------------------------------


%----------------------------------------------------------------------------------------
%	CUADRO COLOR
%----------------------------------------------------------------------------------------
\usepackage{framed, color}
\definecolor{shadecolor}{rgb}{0.74, 0.83, 0.9}

%----------------------------------------------------------------------------------------
%	Identacion
%----------------------------------------------------------------------------------------

\setlength{\parindent}{0pt}


%----------------------------------------------------------------------------------------
%	Encabezado y pie de pagina
%----------------------------------------------------------------------------------------

\usepackage{fancyhdr}
\usepackage{etoolbox}

%\newcommand{\fecha}{03/03/2016}

%\fancyhf{}
% \fancyhead[LO, LE]{ \begin{center}\textsc{\large{\n{\azul{PLAN DE PROYECTO}}}} \end{center} \\ Versión: {\large{\azul{\version}}} \\ Referencia: \small{\azul{\%referencia}}}

%{\footnotesize{\scshape\nouppercase{\leftmark}}}}
%\fancyhead[RO, RE]{ Página: {\large{\azul{\thepage}}} \\ Fecha: {\azul{\today}}}
% \renewcommand{\headrulewidth}{0.4pt}

%\newcommand{\headrulecolor}[1]{\patchcmd{\headrule}{\hrule}{\color{#1}\hrule}{}{}}
% \fancyfoot[LO,LE]{\vspace*{0.05cm}\includegraphics[width=2cm,height=2cm,% keepaspectratio]{logo.png}}
% 
% \fancyfoot[C]{ \\[10pt] \textsc{Soraus} \\ \textsc{Proyecto BaicUAM} }
% \fancyfoot[RO,RE]{\vspace*{0.05cm}\includegraphics[width=2cm,height=2cm,% keepaspectratio]{baicuam.jpg}}

%----------------------------------------------------------------------------------------
%	Incluir pdf
%----------------------------------------------------------------------------------------

\usepackage{pdfpages}
\usepackage{sectsty}
\sectionfont{\fontsize{17}{14}\selectfont}


%----------------------------------------------------------------------------------------
%	Nota al pie de pagina
%----------------------------------------------------------------------------------------

\renewcommand{\thefootnote}{\arabic{footnote}} 


%----------------------------------------------------------------------------------------
%	Sub sub sub seccion
%----------------------------------------------------------------------------------------

\usepackage{titlesec}

\titleclass{\subsubsubsection}{straight}[\subsection]

\newcounter{subsubsubsection}[subsubsection]
\renewcommand\thesubsubsubsection{\thesubsubsection.\arabic{subsubsubsection}}
\renewcommand\theparagraph{\thesubsubsubsection.\arabic{paragraph}} % optional; useful if paragraphs are to be numbered

\titleformat{\subsubsubsection}
  {\normalfont\normalsize\bfseries}{\thesubsubsubsection}{1em}{}
\titlespacing*{\subsubsubsection}
{0pt}{3.25ex plus 1ex minus .2ex}{1.5ex plus .2ex}

\makeatletter
\renewcommand\paragraph{\@startsection{paragraph}{5}{\z@}%
  {3.25ex \@plus1ex \@minus.2ex}%
  {-1em}%
  {\normalfont\normalsize\bfseries}}
\renewcommand\subparagraph{\@startsection{subparagraph}{6}{\parindent}%
  {3.25ex \@plus1ex \@minus .2ex}%
  {-1em}%
  {\normalfont\normalsize\bfseries}}
\def\toclevel@subsubsubsection{4}
\def\toclevel@paragraph{5}
\def\toclevel@paragraph{6}
\def\l@subsubsubsection{\@dottedtocline{4}{7em}{4em}}
\def\l@paragraph{\@dottedtocline{5}{10em}{5em}}
\def\l@subparagraph{\@dottedtocline{6}{14em}{6em}}
\makeatother

